% some shortcuts that simplify live-TeXing using XY.

% Forces all XY entries to be typeset with displaymath, which is much more
% common for me
\everyentry={\displaystyle}

% Short exact sequences: write
% \shortexact[f][g]{A}{B}{C}, for:
%
%		 f    g
% 0 -> A -> B -> C -> 0,
\DeclareDocumentCommand{\shortexact}{O{} O{} mmmm}{
\xymatrix{
	0\ar[r] & #3\ar[r]^-{#1} & #4\ar[r]^-{#2} & #5\ar[r] & 0#6
}}
% exactly the same, but for 0 -> A -> B -> C
\DeclareDocumentCommand{\leftexact}{O{} O{} mmmm}{
\xymatrix{
	0\ar[r] & #3\ar[r]^-{#1} & #4\ar[r]^-{#2} & #5 #6
}}
% ... and the same, for A -> B -> C -> 0
\DeclareDocumentCommand{\rightexact}{O{} O{} mmmm}{
\xymatrix{
	{#3}\ar[r]^-{#1} & #4\ar[r]^-{#2} & #5\ar[r] & 0#6
}}

% Double right arrow, which I found myself writing a lot (e.g. equalizer, kernel,
% or cokernel diagrams)
% usage:
% X\dblarrow[r][f][g] & Y
%   f
% X => Y
%   g
% Since LaTeX is parsing a class file, we need to tell it that @ is not part of the
% \ar command, or we get some opaque errors.
\makeatother
\DeclareDocumentCommand{\dblarrow}{O{} O{} O{}}{%
	\ar@<0.4ex>[#1]^-{#2}\ar@<-0.4ex>[#1]_-{#3}%
}
% Note: it would be a useful exercise to figure out how to define this so it can
% be used as \dblarrow[r]^f_g.

% Useful for morphisms in overcategories (aka slice categories), such as vector bundles,
% covering spaces, field extensions, schemes over a base... or just commutative triangles
% Usage: \overtriangle[f][\pi_1][\pi_2]{X_1}{X_2}{B}.
% Note: the last argument is punctuation; if you don't want punctuation, pass it as {}
\DeclareDocumentCommand{\overtriangle}{O{} O{} O{} mmmm}{
\xymatrix@C=0.4cm{
	{#4}\ar[rr]^{#1}\ar[dr]_{#2} && {#5}\ar[dl]^{#3}\\ % comment for cpp. Don't delete
	& {#6 #7}
}}
\makeatletter

% TODO: do I want these going in the other direction?
% Source: http://tug.org/pipermail/xy-pic/2001-July/000015.html
\newcommand{\pullbackcorner}[1][dr]{\save*!/#1+1.2pc/#1:(1,-1)@^{|-}\restore}
\newcommand{\pushoutcorner}[1][dr]{\save*!/#1-1.2pc/#1:(-1,1)@^{|-}\restore}


% TODO more? Especially pullback or pushout squares.

% Mathematicians have lots of fancy fonts

% \mathbb -- notable sets
% TODO: would be nice to allow people to override \mathbb with \mathbf
\newcommand{\A}{\mathbb A} % affine space
\newcommand{\C}{\mathbb C} % complex numbers
\newcommand{\D}{\mathbb D} % unit disc inside \C
\newcommand{\E}{\mathbb E} % expected value, family of operads
\newcommand{\F}{\mathbb F} % finite fields
\newcommand{\G}{\mathbb G} % additive/multiplicative groups
\AtBeginDocument{ % some fonts redefine this (e.g. charter)
	\renewcommand{\H}{\mathbb H} % quaternions, upper half-plane
}
\newcommand{\N}{\mathbb N} % natural numbers
\renewcommand{\P}{\mathbb P} % probability, projective space
\newcommand{\Q}{\mathbb Q} % rational numbers
\newcommand{\R}{\mathbb R} % real numbers
\newcommand{\Sph}{\mathbb S} % sphere spectrum
\newcommand{\T}{\mathbb T} % circle group
\newcommand{\Z}{\mathbb Z} % integers
\newcommand{\RP}{\mathbb{RP}} % real projective space
\newcommand{\CP}{\mathbb{CP}} % complex projective space

% \mathcal -- lots of different things
\newcommand{\cA}{\mathcal A} % Steenrod algebra, etc.
\newcommand{\cM}{\mathcal M} % moduli space

% \mathfrak -- Lie algebras, open covers, prime ideals
\newcommand{\p}{\mathfrak p} % prime ideal
\newcommand{\q}{\mathfrak q} % another prime ideal
\newcommand{\m}{\mathfrak m} % maximal ideal
\newcommand{\fg}{\mathfrak g} % general Lie algebra
\newcommand{\gl}{\mathfrak{gl}} % general linear Lie algebra
\renewcommand{\sl}{\mathfrak{sl}} % special linear
\renewcommand{\sp}{\mathfrak{sp}} % symplectic
\newcommand{\fo}{\mathfrak o} % orthogonal
\newcommand{\so}{\mathfrak{so}} % special orthogonal
\newcommand{\fu}{\mathfrak u} % unitary
\newcommand{\su}{\mathfrak{su}} % special unitary
\newcommand{\fU}{\mathfrak U} % open covers, à la Bott-Tu

% \mathrm -- usually Lie groups
\newcommand{\GL}{\mathrm{GL}} % general linear
\newcommand{\SL}{\mathrm{SL}} % special linear
\AtBeginDocument{ % redefined by mathdesign
	\renewcommand{\O}{\mathrm O} % orthogonal
}
\newcommand{\SO}{\mathrm{SO}} % special orthogonal
\newcommand{\U}{\mathrm U} % unitary
\newcommand{\SU}{\mathrm{SU}} % special unitary
\newcommand{\Sp}{\mathrm{Sp}} % symplectic
\newcommand{\Spin}{\mathrm{Spin}} % spin
\newcommand{\PGL}{\mathrm{PGL}} % projective general linear
\newcommand{\PSL}{\mathrm{PSL}} % projective special linear

% \mathscr -- usually sheaves
\newcommand{\sF}{\mathscr F} % sheaf
\newcommand{\sG}{\mathscr G} % sheaf
\newcommand{\sH}{\mathscr H} % sheaf
\newcommand{\sI}{\mathscr I} % sheaf of ideals, index category
\newcommand{\sL}{\mathscr L} % line bundle
\newcommand{\sM}{\mathscr M} % quasicoherent sheaf
\newcommand{\sO}{\mathscr O} % ring of functions

% \mathsf -- categories
\newcommand{\cat}{\mathsf}
% The user can redefine \cat to be something else (e.g. mathbf). However, I'd also
% like them to be able to redefine things like Set, Grp, and so forth without having
% to use AtBeginDocument.
% TODO: do I even need this AtBeginDocument here...?
\AtBeginDocument{
	\newcommand{\fC}{\cat C}
	\newcommand{\fD}{\cat D}
	\newcommand{\Set}{\cat{Set}} % sets
	\newcommand{\Grp}{\cat{Grp}} % groups
	\newcommand{\Gpd}{\cat{Gpd}} % groupoids
	\newcommand{\Ab}{\cat{Ab}} % abelian groups
	\newcommand{\Ring}{\cat{Ring}} % rings
	\newcommand{\Mod}{\cat{Mod}} % modules (over a ring)
	\newcommand{\Alg}{\cat{Alg}} % algebras (over a ring)
	\newcommand{\Vect}{\cat{Vect}} % vector spaces (over a field)
	\def\Top{\cat{Top}} % topological space (sometimes already defined, e.g. by kpfonts)
	% TODO what other categories of topological/geometric objects do I need?
	\newcommand{\LocRing}{\cat{LocRing}} % locally ringed spaces
	\newcommand{\AffSch}{\cat{AffSch}} % affine schemes
	\newcommand{\Sch}{\cat{Sch}} % schemes
	\newcommand{\Man}{\cat{Man}} % manifolds
	\newcommand{\Fun}{\cat{Fun}} % functor categories
}

% Setting up some theorem environments
%
% The "exampx" and similar provisional environments are in use so that I can
% append a marker at the end of examples, remarks, claims, facts, and notes.
% (Of course, it would be good for me to determine how to do this automatically.)
% Source: http://tex.stackexchange.com/a/32394/
%
\newtheorem{thm}[equation]{Theorem}
\newtheorem{lem}[equation]{Lemma}
\newtheorem{cor}[equation]{Corollary}
\newtheorem{prop}[equation]{Proposition}
\theoremstyle{definition}
\newtheorem{ex}[equation]{Exercise}
\newtheorem{exampx}[equation]{Example}
\newtheorem{defn}[equation]{Definition}
\newtheorem{claim}[equation]{Claim}
\theoremstyle{remark}
\newtheorem{remx}[equation]{Remark}
\newtheorem*{fctx}{Fact}
\newtheorem*{notex}{Note}
%
%
\newcommand{\exampleQED}{\smash\adfhalfleftarrowhead}
\newenvironment{exm}
  {\pushQED{\qed}\renewcommand{\qedsymbol}{\exampleQED}\exampx}
  {\popQED\endexampx}
\newenvironment{rem}
  {\pushQED{\qed}\renewcommand{\qedsymbol}{\exampleQED}\remx}
  {\popQED\endremx}
\newenvironment{fct}
  {\pushQED{\qed}\renewcommand{\qedsymbol}{\exampleQED}\fctx}
  {\popQED\endfctx}
\newenvironment{note}
  {\pushQED{\qed}\renewcommand{\qedsymbol}{\exampleQED}\notex}
  {\popQED\endnotex}
%
% Another possibility: "definition-theorems" as used in universal properties
% I will probably also want starred theorems (propositions, subsections, etc.).

% there will be more files here.

\numberwithin{equation}{section}

\newcommand{\term}{\emph}

% Often, I find myself making a theorem, definition, etc. that's purely a combination
% of statements, either bulleted or numbered. In that case, using \hfill typesets each
% item uniformly. It would probably be best to make this into a custom enumitem
% environment, especially if I want to refer to specific items (e.g. 12.2.1 inside
% Theorem 12.2).
% Usage: \begin{comp}{thm}{enumerate}
%			\item
% 		 \end{comp}
% sets up an enumerate environment inside a theorem.
\NewDocumentEnvironment{comp}{mm}{%
	\csname #1\endcsname\hfill
	\csname #2\endcsname
}{
	\csname end#2\endcsname
	\csname end#1\endcsname
}

% other shortcuts I use for live-TeXing
\newcommand{\vp}{\varphi}
\newcommand{\e}{\varepsilon}
\newcommand{\inj}{\hookrightarrow}
\newcommand{\surj}{\twoheadrightarrow}
\newcommand{\id}{\mathrm{id}}
\newcommand{\pt}{\mathrm{pt}}
% Use \ud for things that need additional space, e.g. f(x) dx (in integrals or for
% differential forms). \d is for things that don't (e.g. after a wedge, or just dx on
% its own).
\newcommand{\ud}{\,\mathrm{d}}
\renewcommand{\d}{\mathrm d} % TODO \AtBeginDocument
% TODO accept optional argument for higher derivatives
\newcommand{\dfr}[2]{\frac{\mathrm d #1}{\mathrm d #2}}
\newcommand{\pfr}[2]{\frac{\partial #1}{\partial #2}}

% This allows \paren{...} to replace \left(...\right) (and similarly for \bkt). For
% \ang, \set, \abs, and \norm, I find myself using autoexpansion less often.
% This code, along with some other shortcuts, is duplicated in my problem set template;
% perhaps I should have them include a common file?
\DeclarePairedDelimiter\paren{(}{)}
\DeclarePairedDelimiter\ang{\langle}{\rangle}
\DeclarePairedDelimiter\abs{\lvert}{\rvert}
\DeclarePairedDelimiter\norm{\lVert}{\rVert}
\DeclarePairedDelimiter\bkt{[}{]}
\DeclarePairedDelimiter\set{\{}{\}}
% Swap paren* and paren, etc., so that the normal version resizes by default.
% Meanwhile, one can use \paren*[\Big]{...} to customize the size easily.
\makeatletter
	\let\oldparen\paren
	\def\paren{\@ifstar{\oldparen}{\oldparen*}}
	\let\oldbkt\bkt
	\def\bkt{\@ifstar{\oldbkt}{\oldbkt*}}
\makeatother

% This allows a"1 as a shortcut for a^{(1)} and a"{10} as a shortcut for a^{(10)}.
\catcode`\"=13
\newcommand{"}[1]{^{(#1)}}

% Now, for a bunch of field-specific commands. I may split these into separate files

% Algebra
\DeclareMathOperator{\Ann}{Ann}
\DeclareMathOperator{\Cliff}{Cliff}
\DeclareMathOperator{\coker}{coker}
\DeclareMathOperator{\End}{End}
\DeclareMathOperator{\Ext}{Ext}
\DeclareMathOperator{\Frac}{Frac}
\DeclareMathOperator{\Gal}{Gal}
\DeclareMathOperator{\Hom}{Hom}
\renewcommand{\Im}{\operatorname{Im}} % also complex analysis
\DeclareMathOperator{\ker}{ker}
\DeclareMathOperator{\Mat}{Mat}
\newcommand{\op}{^{\cat{op}}} % for the opposite of a category
\DeclareMathOperator{\sign}{sign}
\DeclareMathOperator{\spa}{span}
\DeclareMathOperator{\Stab}{Stab}
\DeclareMathOperator{\Sym}{Sym}
\newcommand{\T}{^{\mathrm T}} % tranpose TODO where does the \! go?
\DeclareMathOperator{\Tor}{Tor}
\DeclareMathOperator{\bl}{--} % maybe I can define this more cleanly

% Algebraic Geometry
\DeclareMathOperator{\Proj}{Proj}
\DeclareMathOperator{\QCoh}{QCoh}
\DeclareMathOperator{\res}{res} % also complex analysis
\DeclareMathOperator{\Spec}{Spec}

% Algebraic Topology
\newcommand{\Gr}{\mathrm{Gr}} % Grassmannian -- perhaps move to letters.tex
\newcommand{\hdr}{H_{\text{\rm dR}}}
\newcommand{\KO}{\mathit{KO}} % should I add reduced K and KO?
\newcommand{\wH}{\widetilde H}

% Complex analysis
% note: Re and Im changes are technically style changes, but almost everyone uses this notation
\renewcommand{\Re}{\operatorname{Re}}

% Topology
\DeclareMathOperator{\codim}{codim}
\DeclareMathOperator{\crit}{crit}
\DeclareMathOperator{\curl}{curl}
\renewcommand{\div}{\operatorname{div}}
\DeclareMathOperator{\ind}{ind}
\DeclareMathOperator{\supp}{supp}

% TODO: to be continued
