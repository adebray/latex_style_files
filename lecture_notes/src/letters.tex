% Mathematicians have lots of fancy fonts
% TODO: Hood has a way to define a whole bunch of these at once

% \mathbb -- notable sets
% TODO: would be nice to allow people to override \mathbb with \mathbf
\newcommand{\A}{\mathbb A} % affine space
\newcommand{\C}{\mathbb C} % complex numbers
\newcommand{\D}{\mathbb D} % unit disc inside \C
\newcommand{\E}{\mathbb E} % expected value, family of operads
\newcommand{\F}{\mathbb F} % finite fields
\newcommand{\G}{\mathbb G} % additive/multiplicative groups
\AtBeginDocument{ % some fonts redefine this (e.g. charter)
	% Note: if you want to write Erdős as Erd\H os, you'll have to use \oldH.
	% If you use [utf8]{inputenc}, this can still be a problem.
	\let\oldH\H
	\renewcommand{\H}{\mathbb H} % quaternions, upper half-plane
}
\newcommand{\N}{\mathbb N} % natural numbers
\renewcommand{\P}{\mathbb P} % probability, projective space
\newcommand{\Q}{\mathbb Q} % rational numbers
\newcommand{\R}{\mathbb R} % real numbers
\newcommand{\Sph}{\mathbb S} % sphere spectrum
\newcommand{\T}{\mathbb T} % circle group
\newcommand{\Z}{\mathbb Z} % integers
\newcommand{\RP}{\mathbb{RP}} % real projective space
\newcommand{\CP}{\mathbb{CP}} % complex projective space

% \mathcal -- lots of different things
\newcommand{\cA}{\mathcal A} % Steenrod algebra, etc.
\newcommand{\cB}{\mathcal B} % bundles of frames
\newcommand{\cG}{\mathcal G} % groupoid
\newcommand{\cH}{\mathcal H} % Hilbert space
\newcommand{\cM}{\mathcal M} % moduli space
\newcommand{\cX}{\mathcal X} % vector fields, tangential structures

% \mathfrak -- Lie algebras, open covers, prime ideals
\newcommand{\p}{\mathfrak p} % prime ideal
\newcommand{\q}{\mathfrak q} % another prime ideal
\newcommand{\m}{\mathfrak m} % maximal ideal
\newcommand{\fg}{\mathfrak g} % general Lie algebra
\newcommand{\gl}{\mathfrak{gl}} % general linear Lie algebra
\renewcommand{\sl}{\mathfrak{sl}} % special linear
\renewcommand{\sp}{\mathfrak{sp}} % symplectic
\newcommand{\fo}{\mathfrak o} % orthogonal
\newcommand{\so}{\mathfrak{so}} % special orthogonal
\newcommand{\fu}{\mathfrak u} % unitary
\newcommand{\su}{\mathfrak{su}} % special unitary
\newcommand{\fU}{\mathfrak U} % open covers, à la Bott-Tu

% \mathrm -- usually Lie groups
\newcommand{\GL}{\mathrm{GL}} % general linear
\newcommand{\SL}{\mathrm{SL}} % special linear
\AtBeginDocument{ % redefined by mathdesign
	\renewcommand{\O}{\mathrm O} % orthogonal
}
\newcommand{\SO}{\mathrm{SO}} % special orthogonal
\newcommand{\U}{\mathrm U} % unitary
\newcommand{\SU}{\mathrm{SU}} % special unitary
\newcommand{\Sp}{\mathrm{Sp}} % symplectic
\newcommand{\Spin}{\mathrm{Spin}} % spin
\newcommand{\Pin}{\mathrm{Pin}} % pin
\newcommand{\PGL}{\mathrm{PGL}} % projective general linear
\newcommand{\PSL}{\mathrm{PSL}} % projective special linear

% The uppercase version is for mathmode or the beginning of a sentence in textmode.
% The lowercase version is for the middle of a sentence in textmode.
% \Pinp for pin+, \Pinm for pin-
% TODO: it would be "fun" to make a macro for these kinds of definitions
\newcommand{\Spinc}{\relax\ifmmode\Spin^c\else Spin\textsuperscript{$c$}\xspace\fi}
\newcommand{\spinc}{spin\textsuperscript{$c$}\xspace}
\newcommand{\Pinc}{\relax\ifmmode\Pin^c\else Pin\textsuperscript{$c$}\xspace\fi}
\newcommand{\pinc}{pin\textsuperscript{$c$}\xspace}
\newcommand{\Pinp}{\relax\ifmmode\Pin^+\else Pin\textsuperscript{$+$}\xspace\fi}
\newcommand{\pinp}{pin\textsuperscript{$+$}\xspace}
\newcommand{\Pinm}{\relax\ifmmode\Pin^-\else Pin\textsuperscript{$-$}\xspace\fi}
\newcommand{\pinm}{pin\textsuperscript{$-$}\xspace}

% \mathscr -- usually sheaves
\newcommand{\sF}{\mathscr F} % sheaf
\newcommand{\sG}{\mathscr G} % sheaf
\newcommand{\sH}{\mathscr H} % sheaf
\newcommand{\sI}{\mathscr I} % sheaf of ideals, index category
\newcommand{\sL}{\mathscr L} % line bundle
\newcommand{\sM}{\mathscr M} % quasicoherent sheaf
\newcommand{\sO}{\mathscr O} % ring of functions

% \mathsf -- categories
% TODO: at least provide a flag allowing for, e.g., categories with \mathcal
\newcommand{\cat}{\mathsf}
% The user can redefine \cat to be something else (e.g. mathbf). However, I'd also
% like them to be able to redefine things like Set, Grp, and so forth without having
% to use AtBeginDocument.
% TODO: do I even need this AtBeginDocument here...?
\newcommand{\newcat}[2]{\newcommand{#1}{\cat{#2}}}
\AtBeginDocument{
	\newcat{\fC}{C}
	\newcat{\fD}{D}
	\newcat{\Set}{Set} % sets
	\newcat{\Grp}{Grp} % groups
	\newcat{\Gpd}{Gpd} % groupoids
	\newcat{\Ab}{Ab} % abelian groups
	\newcat{\Ring}{Ring} % rings
	\newcat{\Mod}{Mod} % modules (over a ring)
	\newcat{\Alg}{Alg} % algebras (over a ring)
	\newcat{\Vect}{Vect} % vector spaces (over a field)
	\newcat{\sVect}{sVect} % super-vector-spaces
	\def\Top{\cat{Top}} % topological space (sometimes already defined, e.g. by kpfonts)
	% TODO what other categories of topological/geometric objects do I need?
	\newcat{\LocRing}{LocRing} % locally ringed spaces
	\newcat{\AffSch}{AffSch} % affine schemes
	\newcat{\Sch}{Sch} % schemes
	\newcat{\sSet}{sSet} % simplicial sets
	\newcat{\Man}{Man} % manifolds
	\newcat{\Fun}{Fun} % functor categories
	\newcat{\Rep}{Rep} % representations
}
