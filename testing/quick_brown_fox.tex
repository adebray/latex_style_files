% The purpose of this document is to test every aspect of my class file in as concise a manner as possible.
% TODO a lot of it is missing...
\documentclass{../pset_d}
\newbbsym{\P}{P}

\course{Math 1729}
\psnum{12}
\author{Arun Debray}
\date{\today}

\begin{document}
\maketitle

\begin{enumerate}
\item %1
The quick brown fox jumps over the lazy dog.
\begin{thm}
If \(\iota:\Z\inj\Q\) is the canonical injection, then \(\iota\) is an epimorphism which is not surjective.
\end{thm}
Over \(\C\), all irreducible polynomials have degree \(1\), and over \(\R\), they all have degree \(1\) or \(2\).
However, over \(\F_p\), they may have any positive degree. This makes life in \(\P\F_p\) more interesting.
\begin{lem}
There is a bijection \(\N\to\N\times\N\).
\end{lem}
\begin{claim}
\(\Q\) is dense in \(\R\).
\end{claim}
\begin{claim*}
\(\overline\Q\) is dense in \(\C\).
\end{claim*}
The derivative is defined by
\begin{equation}
\label{deriv}
	\dfr{f}{x} = \lim_{\e\to 0} \frac{f(x+\e) - f(x)}{\e},
\end{equation}
and the partial derivative by
\begin{equation}
	\pfr{\vp}{x_i} = \lim_{\e\to 0} \frac{\vp(\vec x+\e \vec\alpha_i) - \vp(\vec x)}{\e}.
\end{equation}
Notice the similarities to \eqref{deriv}.
\item And now we test delimiters.
\[\int_a^b f(x)\ud x = \lim_{n\to\infty}\paren*[\Bigg]{\paren{\frac{b-a}{n}}\sum_{j=1}^n f\paren*{x_j^*}}.\]
In one dimension,
\begin{equation}
\abs*{\sum_{i=1}^n x_i} \le \sum_{i=1}^n \abs{x_i}.
\end{equation}
In multiple dimensions,
\begin{equation}
\norm*{\sum_{i=1}^n \vec x_i} \le \sum_{i=1}^n\norm{\vec x_i}.
\end{equation}
\end{enumerate}
\end{document}
