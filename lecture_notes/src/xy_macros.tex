% Some shortcuts that simplify live-TeXing using XY.
%
% Forces all XY entries to be typeset with displaymath, which is much more
% common for me
\everyentry={\displaystyle}
%
% Short exact sequences: write
% \shortexact[f][g]{A}{B}{C}, for:
%
%		 f    g
% 0 -> A -> B -> C -> 0,
\DeclareDocumentCommand{\shortexact}{O{} O{} mmmm}{
\xymatrix{
	0\ar[r] & #3\ar[r]^-{#1} & #4\ar[r]^-{#2} & #5\ar[r] & 0#6
}}
% exactly the same, but for 0 -> A -> B -> C
\DeclareDocumentCommand{\leftexact}{O{} O{} mmmm}{
\xymatrix{
	0\ar[r] & #3\ar[r]^-{#1} & #4\ar[r]^-{#2} & #5 #6
}}
% ... and the same, for A -> B -> C -> 0
\DeclareDocumentCommand{\rightexact}{O{} O{} mmmm}{
\xymatrix{
	{#3}\ar[r]^-{#1} & #4\ar[r]^-{#2} & #5\ar[r] & 0#6
}}
%
% Double right arrow, which I found myself writing a lot (e.g. equalizer, kernel,
% or cokernel diagrams)
% usage:
% X\dblarrow[r][f][g] & Y
%   f
% X => Y
%   g
% Since LaTeX is parsing a class file, we need to tell it that @ is not part of the
% \ar command, or we get some opaque errors.
\makeatother
\DeclareDocumentCommand{\dblarrow}{O{} O{} O{}}{%
	\ar@<0.4ex>[#1]^-{#2}\ar@<-0.4ex>[#1]_-{#3}%
}
% Note: it would be a useful exercise to figure out how to define this so it can
% be used as \dblarrow[r]^f_g.
%
% Useful for morphisms in overcategories (aka slice categories), such as vector bundles,
% covering spaces, field extensions, schemes over a base... or just commutative triangles
% Usage: \overtriangle[f][\pi_1][\pi_2]{X_1}{X_2}{B}.
% Note: the last argument is punctuation; if you don't want punctuation, pass it as {}
\DeclareDocumentCommand{\overtriangle}{O{} O{} O{} mmmm}{
\xymatrix@C=0.4cm{
	{#4}\ar[rr]^{#1}\ar[dr]_{#2} && {#5}\ar[dl]^{#3}\\ % comment for cpp. Don't delete
	& {#6 #7}
}}
\makeatletter
%
% TODO: do I want these going in the other direction?
% Source: http://tug.org/pipermail/xy-pic/2001-July/000015.html
\newcommand{\pullbackcorner}[1][dr]{\save*!/#1+1.2pc/#1:(1,-1)@^{|-}\restore}
\newcommand{\pushoutcorner}[1][dr]{\save*!/#1-1.2pc/#1:(-1,1)@^{|-}\restore}
%
% TODO more? Especially pullback or pushout squares.
%
% An xymatrix environment wrapped in gathered to ensure its equation number is vertically centered
\newcommand{\gathxy}[2][]{%
\begin{gathered}
\xymatrix#1{#2}
\end{gathered}
}
