\begin{quote}\textit{
	%``I'm not going to give the Rees construction, but at least I mentioned it.''
	``This is a proof by intimidation.''
}\end{quote}
We're going to be doing some calculus in the algebraic geometry setting over the next few weeks, starting with
differentials and the language to say all the things we need to say. In order to talk about differentials, we need
to spend some time talking about duals. To do this, we need to make one caveat.

Let \(R\) be a ring and \(M\) be an \(R\)-module, so we have an associated quasicoherent sheaf \(\sM = \Delta(M)\),
which is an \(\sO_X\)-module. We can form the dual module \(M^\vee = \Hom_R(M,R)\), and therefore this suggests
taking a dual sheaf \(\sM^\vee = \shom_{\sO_X}(\sM,\sO_X)\). This is the correct notion if \(\sM\) is locally free
(akin to vector bundles), but not in general.

Suppose \(\sM\) is the skyscraper sheaf at \(0\) on \(\A^n\), which is the localization of the \(k[x,y]\)-module
\(M = k\), where the action of a \(p\in k[x,y]\) is multiplying by \(p(0)\). Thus, \(M\) is torsion, so \(M^\vee =
\Hom_{k[x,y]}(k,k[x,y]) = 0\): there are no maps from a torsion module to a free one.

There's always a map from \(M\) to its double-dual \(M\to M^{\vee\vee}\) given by sending \(m\in M\) to the map
\((\vp\mapsto\vp(m))\), which is a map \(\Hom_R(M,R)\to R\), i.e.\ an element of \(M^{\vee\vee}\). This is a
natural map, but it doesn't have to be an isomorphism: for the module that induces the skyscraper sheaf, it's zero.
\begin{defn}
An \(R\)-module \(M\) is \term{reflexive} if the natural map \(M\to M^{\vee\vee}\) is an isomorphism.
\end{defn}
The modules corresponding to free and locally free sheaves are reflexive. This is the sense in which the dual is
actually dual; in general, it doesn't behave like you might expect.
\begin{exm}
Let's consider a less silly example than the skyscraper sheaf: \(\m_0 = (x,y)\subset k[x,y]\) is an ideal, and
therefore a \(k[x,y]\)-module. Then, \(\m_0^\vee = \Hom_{k[x,y]}(\m_0,k[x,y])\). If \(\vp:\m_0\to k[x,y]\) is a
homomorphism, then \(\alpha = \vp(x)\) and \(\beta = \vp(y)\) determine the homomorphism. However, they're not
linearly independent: we need \(y\alpha = x\beta\), since they're both \(\vp(x,y)\). Therefore, \(\alpha = fx\) and
\(\beta = fy\) for some \(f\in k[x,y]\), and more generally \(\vp(r) = fr\): \(f = \alpha/x = \beta/y\in k[x,y]\).
Thus, \(\Hom_{k[x,y]}(\m_0,k[x,y]) = k[x,y]\). And certainly \((k[x,y])^\vee = k[x,y]\), since it's free over
itself, and the canonical map \(\m_0\to\m_0^{\vee\vee} = k[x,y]\) is the inclusion.
\end{exm}
Thus, torsion-free does not imply reflexive! Duals are weird: they forget information.

The point of this is that to construct the ring of total sections, we took the dual, so we have to be careful. Let
\(M\) be an \(R\)-module, so that \(\Sym_R M\) is an \(R\)-algebra, given by \(R\oplus M\oplus\Sym^2 M\oplus\Sym^3
M\oplus\dotsb\).
\begin{defn}
If \(R\) is a ring, an \(R\)-algebra \(M\) is an \term{augmented \(R\)-algebra} if there's a \term{augmentation
map} \(\e:M\to R\) that's a section for the \(R\)-algebra map \(\vp:R\to M\), i.e.\ \(\e\circ\vp = \id\).
\(\ker(\e)\) is known as the \term{augmentation ideal}.
\end{defn}
For example, quotienting out by all terms of positive degree in \(\Sym_R M\) defines an augmentation map
\(\e:\Sym_R M\to R\), so \(\Sym_R M\) is an augmented \(R\)-algebra, and the augmentation ideal is (generated by)
the terms of positive degree.

Geometrically, taking \(\Spec\) turns everything around: if \(X = \Spec R\) and \(Y = \Spec\Sym_R M\), then the
data that \(\Sym_R M\) is an \(R\)-algebra gives us a map \(Y\to X\), and \(\e^*:X\to Y\) is a section for this
map. \(\Sym_R M\) is in fact a graded \(R\)-algebra, and therefore there's a \(\G_m\)-action on \(Y\).

For example, let \(R = k[x]\) and \(M = k[x]/(x)\), which can be thought of as \(k\) plus an action of \(x\). Let
\(y\) be a generator of \(M\), so that \(\Sym_R M = k[x]\oplus k\cdot y\oplus k\cdot y^2\oplus k\cdot
y^3\oplus\dotsb\cong k[x,y]/(xy)\).

If we take \(\Spec\), this is just the union of the \(x\)- and \(y\)-axes in \(\A_k^2\), projecting down onto
\(\Spec R = \A_k^1\). This isn't quite a vector bundle: the fiber over every point is a \(k\)-vector space, but at
\(0\) it jumps (really, it's a skyscraper over \(0\)), and this is bad.

We can recover \(M\) from the algebraic data of \(\Sym_R M\) as the degree-\(1\) elements, and there is also a way
to do this geometrically. Explicitly, to get the terms of degree-\(1\), take the augmentation ideal and remove all
terms of degree at least \(2\): \(M\cong I/I^2\) as \(R\)-modules.

Alternatively, one could take the ring \(\Sym_R M/I^2\), which is the split square-zero extension \(R\oplus M\), in
the way that we talked about last time (so \(M\) multiplies to \(0\)). The augmentation survives as the map
\(\e:\Sym_R M/I^2\to\Sym_M I \cong R\).

This is equivalent data to what we talked about last time, but is more geometric. The maps of rings induce maps of
schemes \(\Spec R\to\Spec(R\oplus M)\to\Spec\Sym_R M\). Here, \(\Spec(R\oplus M)\) is the first-order neighborhood
of the ``zero section'' in the ``total space`` of \(M\), which is \(\Spec\Sym_R M\). This is because \(I\) cuts out
the zero section, but we're modding out by \(I^2\), which gives us the first-order neighborhood, as with the dual
numbers. In fact, if \(M\) is free of dimension \(n\), \(\Spec(R\oplus M) = \Spec R\times (\Spec
(k[\e]/(\e^2)))^{\oplus n}\); if \(M\) is locally free, then this is true locally.\footnote{More generally, one can
consider things such as \(\Spec(\Sym_R M/I^n)\), which is something people do, though in this context it's not so
useful. The point is that we're going to eventually think of \(I/I^2\) as the conormal bundle. We'll return to
things like this.}
\begin{defn}
With \(R\), \(M\), and \(I\) as in the preceding discussion, let \(X = \Spec R\) and \(Y = \Spec \Sym_R M\). Then,
the \term{conormal sheaf} to \(X\inj Y\) is the sheaf associated to the \(R\)-module \(I/I^2\).
\end{defn}
This seems like it should be the normal sheaf (analogous to the normal bundle), but if you look carefully, this is
really linear functionals, so it is more like a dual space.

For example, suppose \(R = k\), so \(X = \Spec k = \Spec(\Sym_R M/\m)\), where \(\m\) is a maximal ideal of the
symmetric algebra. In this case, \(I/I^2 = \m/\m^2\), which is the cotangent space.

We'll talk about cotangents in order to make conormals make more sense, and hence talk about derivations.

Recall that if \(A\) is a \(B\)-algebra and \(M\) is an \(A\)-module, we make \(A\oplus M\) into a \(B\)-algebra as
a square-zero extension, like last time. Then, the space of derivations is \(\Der_B(A,M) = \Hom_B(A,A\oplus M)\).
These are the \(B\)-linear functions \(\partial:A\to M\) such that \(\partial(fg) = f\partial g + g\partial f\).
These feel like differential operators; for example, \(\pfr{}{x}\in\Der_k(k[x],k[x])\).

If you've been sufficiently Grothendiecized by this class, you should expect some sort of ``universal'' of ``free''
derivation given the \(B\)-algebra structure \(\vp: B\to A\). This will be an \(A\)-module \(\Omega_{A/B}\) along
with a map \(\d:A\to \Omega_{A/B}\).
\begin{defn}
Define the \(A\)-module \(\Omega_{A/B}\), the module of \term{(Kähler) differentials of \(A\) over \(B\)}, to be
the \(A\)-module spanned by elements \(\d a\) for all \(a\in A\) subject to the following relations for all
\(a,a'\in A\):
\begin{itemize}
	\item \(\d a + \d a' = \d(a+a')\),
	\item \(\d(aa') = a\d a' + a'\d a\), and
	\item \(\d(\vp(b)) = 0\).
\end{itemize}
Then, define \(\d:A\to M\), the \term{de Rham differential}, to send \(a\mapsto \d a\).
\end{defn}
The first relation forces \(\d\) to be \(A\)-linear, and the second is the Leibniz rule. The last rule makes this
compatible with the structure of \(B\).

This feels nostalgically like the construction of the tensor product, and so we should expect a universal property.
\begin{prop}
\label{diffuniv}
Let \(M\) be an \(A\)-module and \(\partial:A\to M\) be a derivation. Then, there is a unique derivation
\(\widetilde\partial:\Omega_{A/B}\to M\) such that the following diagram commutes.
\[\xymatrix@R=0.3cm{
	& \Omega_{A/B}\ar@{-->}[dd]_{\widetilde\partial}^{\exists!}\\
	A\ar[ur]^\d\ar[dr]_\partial\\
	& M
}\]
\end{prop}
The construction is to let \(\widetilde\partial(\d a) = \partial a\) and extend \(A\)-linearly. As a consequence,
\(\Hom_A(\Omega_{A/B}, M) = \Der_B(A,M)\) as \(B\)-modules.

For example, \(T_{A/B} = \Omega_{A/B}^\vee = \Hom_A(\Omega_{A/B},A)\) can be thought of as vector fields, because
it's identified with \(\Der_B(A,A)\). Unlike in differential geometry, we're already doing everything
relatively, and so these are ``relative vector fields,'' compatible with the map \(\Spec B\to\Spec A\). If you want
to understand absolute vector fields, you can take \(B = \Z\), since \(\Z\)-linearity is just additivity, which
doesn't tell us anything. But the flexibility of taking something relative (which might be a point) is still very
useful.

An example of \(T_{A/B}\) is \(A = k[x,y]\) as a \(B = k[x]\)-module; then, \(\Der_{k[x]}(k[x,y],k[x,y]) \cong
k[x,y]\), where \(1\mapsto \pfr{}{y}\).

Here's a neat definition, though we haven't earned it.
\begin{defn}
A map \(\Spec B\to\Spec A\) is \term{smooth} if the module of differentials \(\Omega_{A/B}\) induced from this map
is locally free.
\end{defn}
There are many questions here: what does it mean for a module to be locally free? It's the same notion turned
around, so it's free after sufficiently strong localizations. Over affine schemes, this is the same as free, but
this is very far from true in general. Another nice consequence is that smoothness of any scheme is smoothness of
the induced map to \(\Spec\Z\). This probably isn't very enlightening; the next time we return to smoothness, we'll
have the context to appreciate it more.

A vector bundle over a contractible manifold is trivial, which isn't very hard to show. Is the same true in
algebraic geometry? The best example of something ``contractible'' is affine space, right?
\begin{thm}[Serre's conjecture/Quillen-Suslin theorem]
If \(k\) is an algebraically closed field, all vector bundles on \(\A_k^n\) are trivial.
\end{thm}
This is a scary, hard theorem: look at those big names! More seriously, the proof of this theorem was one of the
first major breakthroughs demonstrating the power of algebraic \(K\)-theory. We definitely haven't earned this
theorem.

Anyways, the point is, ``smooth things should have tangent bundles.'' This is a philosophy, but we can just define
it.
\begin{defn}
Let \(A\) be a \(k\)-algebra. Then, the \term{tangent bundle} of \(X = \Spec A\) is \(TX = \Spec(\Sym_k
\Omega_{A/k})\), and the \term{projectivized tangent bundle} is \(\P(TX) = \Proj(\Sym_k\Omega_{A/k})\).
\end{defn}
\(TX\) locally looks like \(X\times\A_k^n\), and \(\P(TX)\) locally looks like \(X\times\P_k^{n-1}\).

The point is that this will be a vector bundle iff \(X\) is smooth. We'll have to unwrap this later. But it
advertises another good fact about algebraic geometry: from the beginning, we care about singularities, because
rings have singularities. To understand smoothness in a geometric sense, we need calculus, which is why we're
talking about differentials.

We defined \(\Omega_{A/B}\) with a lot of generators and a lot of relations; if we have generators and relations
for \(A\) as a \(B\)-algebra, we can simplify this. In particular, we can always assume \(A\cong B[x_i: i\in
I]/(r_j: j\in J)\). In this case, \(\Omega_{A/B}\) is much simpler:
\[\Omega_{A/B}\cong\paren{\bigoplus_{i\in I} \d x_i}/(\d r_j: j\in J).\]
The de Rham differential of a relation is given by expanding \(A\)-linearly and using the Leibniz rule, e.g.\
\(\d(xy) = x\ud y + y\ud x\). This construction of \(\Omega_{A/B}\) makes it a little more apparent that
\(\Omega_{A/B}\) is a ``linearization'' of the structure of \(A\). It also makes some nice properties apparent.
\begin{cor}
If \(A\) is a finitely generated (resp.\ finitely presented) \(B\)-algebra, then \(\Omega_{A/B}\) is a finitely
generated (resp.\ finitely presented) \(A\)-module.
\end{cor}
Now, suppose \(\vp:B\surj A\) is surjective, so \(A\cong B/I\) for an ideal \(I\subset B\); geometrically, we'd
have an inclusion of schemes. Then, \(\Omega_{A/B} = 0\), because every \(a\in A\) is \(\vp(b)\) for some \(b\in
B\), so \(\d a = \d(\vp(b)) = 0\).

Localizations (open subsets) also don't have any relative differentials: if \(f/g\in S^{-1}B\), then
\[\partial\paren{\frac fg} = \frac{g\partial f - f\partial g}{g^2} = 0,\]
because \(f,g\in B\), so \(\partial f = \partial g = 0\). Hence, \(\Omega_{S^{-1}B/B} = 0\).
\begin{exm}
Suppose \(A = k[x,y]/(y^2 - x^3 + x)\), which corresponds to the elliptic curve \(y^2 = x^3 - x\). Then,
\(\Omega_{A/k} = (A\ud x\oplus A\ud y) / (2y\ud y = (3x^2+1)\ud x)\).

Is this smooth? In other words, is \(\Omega_{A/k}\) locally free? If \(y\ne 0\), then \(\d x\) generates, and if
\(3x^2 + 1\ne 0\), then \(\d y\) is a generator. If \(y = 0\), then \(x^3 - x = 0\), so \(x = 0,\pm 1\), and
therefore \(3x^2+1\ne 0\). Thus, these cover everything, so \(\Omega_{A/k}\) is a line bundle (locally free of rank
\(1\))!\footnote{This is actually a trivial line bundle: one can write down a nowhere-vanishing differential.} In
particular, this curve is smooth.
\end{exm}
\begin{exm}
Our favorite singular curve (well, should be singular) is \(A = k[x,y]/(xy)\): the singularity is at the origin.
Then, \(\Omega_{A/k} = (A\ud x\oplus A\ud y)/(x\ud y + y\ud x)\). Thus, on the \(y\)-axis, \(\d y\) generates, and
on the \(x\)-axis, \(\d x\) generates, but at the origin, we need both of them. Thus, \(\Omega_{A/k}\) isn't
locallty free, so this curve isn't smooth. However, \(\Omega_{A/k}|_0\cong\m_0/\m_0^2\), where \(\m_0\) is the
maximal ideal corresponding to the origin.
\end{exm}
This fact about the fiber is more general: \(\Omega_{A/k}\) is a nice way to put all the contangent spaces
together.
\begin{prop}
Let \(A\) be a \(k\)-algebra and \(x\in\Spec A\) correspond to the maximal ideal \(\m_x\). If \(\m_x\) has residue
field \(k\), then there's an isomorphism \(\delta:\m_x/\m_x^2\to\Omega_{A/k}|_x = \Omega_{A/k}\otimes_A k_x\).
\end{prop}
Geometrically, we're tensoring with the skyscraper sheaf to obtain the fiber.

We're not going to prove this today, for a lack of time. But one can use this to construct a quasicoherent sheaf
\(\widetilde\Omega_{A/k}\), defined by \(\widetilde\Omega_{A/k}(D(f)) = \Omega_{A_f/k}\), so localizing as usual.
We'll also consider more geometric ways of understanding this.

Another useful word is the analogue of a covering space: there's no way to differentiate along the fibers.
\begin{defn}
If \(A\) and \(B\) are \(k\)-algebras, where \(k\) is characteristic zero, then if \(\Omega_{A/B} = 0\), then the
map \(B\to A\) is called \term{étale}.
\end{defn}
There is a definition of étale in positive characteristic, but this isn't the correct definition.
