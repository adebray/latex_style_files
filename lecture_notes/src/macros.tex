#include "xy_macros.tex"
#include "letters.tex"
#include "theorems.tex"

\numberwithin{equation}{section}
%
% This could be considered a stylistic choice, but works much better with how I use
% subsections as subject changes in a lecture.
\setcounter{tocdepth}{1}
%
% These can be changed downstream
\newcommand{\term}{\emph} % e.g. "The \term{trace} is defined to be..."
\newcommand{\latin}{\textit} % e.g. \latin{per se}, \latin{mutatis mutandis}
%
% Often, I find myself making a theorem, definition, etc. that's purely a combination
% of statements, either bulleted or numbered. In that case, using \hfill typesets each
% item uniformly. It would probably be best to make this into a custom enumitem
% environment, especially if I want to refer to specific items (e.g. 12.2.1 inside
% Theorem 12.2).
% Usage: \begin{comp}{thm}{enumerate}
%			\item
% 		 \end{comp}
% sets up an enumerate environment inside a theorem.
\NewDocumentEnvironment{comp}{mm}{%
	\csname #1\endcsname\hfill
	\csname #2\endcsname
}{
	\csname end#2\endcsname
	\csname end#1\endcsname
}
%
% since I find myself using \renewcommand{...}{\operatorname{...}} a lot.
% possible TODO: also define a starred version?
% could also define \ProvideMathOperator but I wouldn't use that at all
\newcommand{\RenewMathOperator}[2]{\renewcommand{#1}{\operatorname{#2}}}
%
% other shortcuts I use for live-TeXing
\newcommand{\vp}{\varphi}
\newcommand{\e}{\varepsilon}
\newcommand{\inj}{\hookrightarrow}
\newcommand{\surj}{\twoheadrightarrow}
\newcommand{\id}{\mathrm{id}}
\newcommand{\pt}{\mathrm{pt}}
\newcommand{\many}[2][\dotsb]{#2 #1 #2} % optional argument for kind of dots
\newcommand{\TFAE}{The following are equivalent}
\newcommand{\TODO}{\textcolor{red}{TODO}}
%
% use a longer \mapsto in display math
\let\shortmapsto\mapsto
\renewcommand{\mapsto}{\mathchoice{\longmapsto}{\shortmapsto}{\shortmapsto}{\shortmapsto}}
%
% Use \ud for things that need additional space, e.g. f(x) dx (in integrals or for
% differential forms). \d is for things that don't (e.g. after a wedge, or just dx on
% its own). I would be interested in unifying them.
% Some packages that I use often (e.g. font packages) overwrite \d, and therefore this
% class needs to define \d after it's overwritten, which is why I use \AtBeginDocument.
% TODO: I want a command for ∂^2f/∂x∂y
\AtBeginDocument{
	\newcommand{\ud}{\,\mathrm{d}}
	\renewcommand{\d}{\mathrm d}
	% accepts optional argument for higher derivatives
	\newcommand{\dfr}[3][]{\frac{\mathrm d^{#1} #2}{\mathrm d #3{}^{#1}}}
	\newcommand{\@dpfr}[3][]{\frac{\partial^{#1} #2}{\partial #3{}^{#1}}}
	\newcommand{\@spfr}[3][]{\partial^{#1} #2 / \partial #3{}^{#1}}
	\newcommand{\pfr}[3][]{\mathchoice{\@dpfr[#1]{#2}{#3}}{\@dpfr[#1]{#2}{#3}}
		{\@spfr[#1]{#2}{#3}}{\@spfr[#1]{#2}{#3}}}
}
%
% This allows \paren{...} to replace \left(...\right) (and similarly for \bkt). For
% \ang, \set, \abs, and \norm, I find myself using autoexpansion less often.
% This code, along with some other shortcuts, is duplicated in my problem set template;
% perhaps I should have them include a common file?
\DeclarePairedDelimiter\paren{(}{)}
\DeclarePairedDelimiter\ang{\langle}{\rangle}
\DeclarePairedDelimiter\abs{\lvert}{\rvert}
\DeclarePairedDelimiter\norm{\lVert}{\rVert}
\DeclarePairedDelimiter\bkt{[}{]}
\DeclarePairedDelimiter\set{\{}{\}}
% Swap paren* and paren, etc., so that the normal version resizes by default.
% Meanwhile, one can use \paren*[\Big]{...} to customize the size easily.
\let\oldparen\paren
\def\paren{\@ifstar{\oldparen}{\oldparen*}}
\let\oldbkt\bkt
\def\bkt{\@ifstar{\oldbkt}{\oldbkt*}}
%
% This allows a"1 as a shortcut for a^{(1)} and a"{10} as a shortcut for a^{(10)}.
\ifdefined\noqu@es
\else
	\AtBeginDocument{\catcode`\"=13}
	\catcode`\"=13
	\newcommand*{"}[1]{^{(#1)}}%
	\catcode`\"12
\fi
%
% Now, for a bunch of field-specific commands.
% TODO: document
% TODO: starred command, perhaps as http://tex.stackexchange.com/a/4388/
\newcommand{\newoperator}[1]{\expandafter\DeclareMathOperator\csname #1\endcsname{\operatorname{#1}}}
%
% Algebra
\newoperator{Ann}
\newoperator{Aut}
\newoperator{Cliff}
\DeclareMathOperator{\chr}{char}
\newoperator{coker}
\newoperator{End}
\newoperator{Ext}
\newoperator{Frac}
\newoperator{Gal}
\newoperator{Hom}
\AtBeginDocument{ % some packages redefine this
	\RenewMathOperator{\Im}{Im} % also complex analysis
}
\newoperator{Mat}
\newcommand{\op}{^{\cat{op}}} % for the opposite of a category
\newoperator{rank}
\newoperator{sign}
\DeclareMathOperator{\spa}{span}
\newoperator{Stab}
\newoperator{Sym}
%TODO better name (conflicts w/ circle group)
%\newcommand{\T}{^{\mathrm T}} % tranpose TODO where does the \! go?
\newoperator{Tor}
\newoperator{tr}
\newcommand{\bl}{\text{--}}
%
% Algebraic Geometry
\newoperator{Proj}
\newoperator{QCoh}
\newoperator{res} % also complex analysis
\newoperator{Spec}
%
% Algebraic Topology
\newcommand{\Gr}{\mathrm{Gr}} % Grassmannian -- perhaps move to letters.tex
\newcommand{\hdr}{H_{\mathrm{dR}}}
\newcommand{\KO}{\mathit{KO}} % should I add reduced K and KO?
\newcommand{\wH}{\widetilde H}
\DeclareMathOperator*{\colim}{colim} % colimit notation in homotopy theory
\DeclareMathOperator*{\holim}{holim} % homotopy limit
\DeclareMathOperator*{\hocolim}{hocolim} % homotopy colimit
%
% Complex analysis
% note: Re and Im changes are technically style changes, but almost everyone uses this notation
\AtBeginDocument{
	\RenewMathOperator{\Re}{Re}
}
\newcommand{\delbar}{\overline\partial}
%
% Topology
\newoperator{codim}
\newoperator{crit}
\newoperator{curl}
\newoperator{Diff}
\RenewMathOperator{\div}{div}
\newoperator{ind}
\newoperator{supp}
%
% to be continued (e.g. a good argmin and argmax)
