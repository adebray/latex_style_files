% The purpose of this document is to test every aspect of my class file in as concise a manner as possible.
% TODO a lot of it is missing...
\documentclass[onesided]{../pset_d}
\newbbsym{\P}{P}

\course{Math 1729}
\psnum{12}
\author{Arun Debray}
\date{\today}

\begin{document}
\maketitle

\begin{enumerate}
\item %1
\label{first}
The quick brown fox jumps over the lazy dog.
\begin{thm}
If \(\iota:\Z\inj\Q\) is the canonical injection, then \(\iota\) is an epimorphism which is not surjective;
nonetheless, writing \(\Z\surj\Q\) would be confusing.
\end{thm}
Over \(\C\), all irreducible polynomials have degree \(1\), and over \(\R\), they all have degree \(1\) or \(2\).
However, over \(\F_p\), they may have any positive degree. This makes life in \(\P\F_p\) more interesting.
\begin{lem}
There is a bijection \(\N\to\N\times\N\).
\end{lem}
\begin{proof}
Count up each diagonal from \((0,0)\), then \((1,0))\) and \((0,1)\), then \((2,0)\), \((1,1)\), and \((0,2)\), and
so on; then, every pair in \(\N\times\N\) will be reached at some finite step.
\end{proof}
\begin{claim}
\(\Q\) is dense in \(\R\).
\end{claim}
\begin{claim*}
\(\overline\Q\) is dense in \(\C\).
\end{claim*}
The derivative is defined by
\begin{equation}
\label{deriv}
	\dfr{f}{x} = \lim_{\e\to 0} \frac{f(x+\e) - f(x)}{\e},
\end{equation}
and the partial derivative by
\begin{equation}
	\pfr{\vp}{x_i} = \lim_{\e\to 0} \frac{\vp(\vec x+\e \vec\alpha_i) - \vp(\vec x)}{\e}.
\end{equation}
Notice the similarities to \eqref{deriv}.
\item[A] And now we test delimiters. \label{second}
\[\int_a^b f(x)\ud x = \lim_{n\to\infty}\paren*[\Bigg]{\paren{\frac{b-a}{n}}\sum_{j=1}^n f\paren*{x_j^*}}.\]
In one dimension,
\begin{equation}
\abs*{\sum_{i=1}^n x_i} \le \sum_{i=1}^n \abs{x_i}.
\end{equation}
In multiple dimensions,
\begin{equation}
\norm*{\sum_{i=1}^n \vec x_i} \le \sum_{i=1}^n\norm{\vec x_i}.
\end{equation}
In an inner product space, \(\norm{\vec x}^2 = \ang{\vec x,\vec x}\). Thus, in \(\ell^2\), if \(\set{x"n}\) is a
sequence converging to \(x\), then \(\ang{x"m-x"n,x"m-x"n}\to 0\) as \(m,n\to\infty\).
\item This item will test that I've set my \verb+itemize+ and \verb+enumerate+ environments correctly: one of the
biggest reasons I wanted to fix my problem set template was that paragraphs in nested enumeration environments
have no indentation or extra space by default, and this makes more textual arguments look bad. I'll also reference
problems~\ref{first} and \ref{second} here.

Paragraphs are no longer indented by default; this is replaced with extra whitespace, which I think looks better.
\begin{enumerate}
	\item This is the next level of indentation. I expect to use it a lot, e.g.\ when one problem comes with
	multiple subparts. As such, I would like it to look good.

	In particular, there should be space between these paragraphs, so that it's easier to see where one part of an
	argument ends and another begins.

	There should also be space between items of the same list.
	\item Here's another item, with some math in it. If \(f\) is a holomorphic function on a disc \(D\) and
	\(\gamma\) is a \(C^1\) loop whose image is in \(D\), then
	\[\oint_\gamma f(z)\ud z = 0.\]
	\item We must go deeper.
	\begin{itemize}
		\item This level will also come up a lot, though I don't expect to go much further. I tend to use bulleted
		lists to organize proofs with a lot of steps that don't depend on each other, e.g.\ to verify that
		something is an abelian group, one typically has to check associativity, commutativity, the presence of an
		identity, and the existence of inverses, but these don't entirely depend on each other.

		Here's another paragraph of the same argument.
		\item And another bullet point, for completeness.
	\end{itemize}
\end{enumerate}
\item Here's a bunch more equations.
\begin{align}
	\label{345}
	3^2 + 4^2 &= 5^2\\
	5^2 + 12^2 &= 13^2\\
	\label{724}
	7^2 + 24^2 &= 25^2\\
	8^2 + 15^2 &= 17^2\\
	9^2 + 40^2 &= 41^2.
\end{align}
Reference to \eqref{345} and to \eqref{724}.
\end{enumerate}
\end{document}
