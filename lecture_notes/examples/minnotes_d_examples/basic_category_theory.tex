\begin{quote}\textit{
	``To this end, we're going to give a crash course in category theory over the next few lectures; the door is
	over there.''
}\end{quote}
Remember that our general agenda is to match algebra and geometry; one way to express this idea is to take the
category of rings and identify it with some category of geometric objects. However, we're going to reverse the
arrows, and we'll get the category of affine schemes. These are some geometric spaces, with a contravariant functor
from affine schemes to rings given by taking the ring of functions and a functor in the opposite direction called
\(\Spec\).

One potential issue is that spaces may not have enough functions, e.g.\ \(\C\P^1\) as a complex manifold only has
constant functions; as such, we'll enlarge our category to a whole category of schemes, which will also have an
algebraic interpretation. Another weird aspect is that functions may take values in varying fields.

Schemes generalize geometry in three different directions: gluing spaces together to ensure we have enough
functions is topology, like making manifolds; functions having varying codomains is useful for arithmetic and
number theory; and allowing for rings with nilpotents feels a little like analysis.

Last time, we defined \(\MSpec(R)\) for a ring \(R\), the set of maximal ideals. It turns out that topology is not
sufficient to understand these spaces; for example, the class of \term{local rings} are those with only one maximal
ideal. There are many such rings, e.g.\ \(\C[x]/(x^n)\), whose maximal ideal is \((x)\). In short, \(\MSpec\)
doesn't see nilpotents.

To any ring \(R\), one can attach the category \(\mathsf{Mod}_R\), whose objects are \(R\)-modules and morphisms
are \(R\)-linear maps (those commuting with the action of \(R\)). This category is one of the more important things
one studies in algebra, and we also want to express them in terms of geometric objects that are related somehow to
\(\Spec R\). This should also help us understand the algebraic properties of \(R\)-modules too.
\subsection*{Crash Course in Categories.}
There's a lot of categorical notions in algebraic geometry; it does strike one as a painful way to start a course,
but hopefully we can get it out of our systems and move on to geometry knowing what we need. This corresponds to
chapters 1 and 2 in the book.

We've seen several examples of categories: sets, groups, rings, etc. The next example is a useful class of
categories.
\begin{defn}
A \term{poset} is a set \(\fS\) and a relation \(\le\) on \(\fS\) that is
\begin{itemize}
	\item \term{reflexive}, so \(x\le x\) for all \(x\in\fS\);
	\item \term{transitive}, so if \(x\le y\) and \(y\le z\), then \(x\le z\); and
	\item \term{antisymmetric}, so if \(x\le y\) and \(y\le x\), then \(x = y\).
\end{itemize}
\(\fS\) has the structure of a category: the objects are the elements of \(\fS\), and \(\Hom(x,y)\) is
\(\set{\pt}\) if \(x\le y\) and is empty otherwise.
\end{defn}
Transitivity means that we have composition, and reflexivity gives us identity maps.

This is an unusual example compared to things like ``the category of all (somethings),'' but is quite useful: a
functor from the poset \(\bullet\to\bullet\) to another category \(\fC\) is a choice of \(A,B\in\fC\) and a map
\(A\to B\); a functor from the poset \(\N\) is the same as an infinite sequence in \(\fC\), and a commutative
diagram is the same as a functor out of the category
\[\xymatrix{
	\bullet\ar[r]\ar[d] & \bullet\ar[d]\\
	\bullet\ar[r] & \bullet
}\]
into \(\fC\).
\begin{exm}
A particularly important example of this: if \(X\) is a topological space, then its open subsets form a poset under
inclusion. Hence, they form a category, called \(\Top(X)\). This category is important for sheaf theory, which we
will say more about later. For example, if \(A\) is an abelian group and \(U\subset X\) is open, then let
\(\sO_A(U)\) denote the abelian group of \(A\)-valued functions on \(U\) (for example, \(A\) might be \(\C\), so
\(\sO_A(U) = C^\infty(U)\)). If \(V\subset U\), then restriction of functions defines a map \(\res_U^V:
\sO_A(U)\to\sO_A(V)\). Since restriction obeys composition, then we've defined a functor \(\sO_A:
\Top(X)\op\to\mathsf{Ab}\) (or perhaps to \(\C\)-algebras, or another category); this is a \term{presheaf of
abelian groups} (or \(\C\)-algebras, etc.).
\end{exm}
To be precise, a \term{presheaf} on \(X\) is a functor out of \(\Top(X)\op\). This is a way of organizing functions
in a way that captures restriction; it will be very useful throughout this class.

Returning to category theory, one of its greatest uses is to capture structure through universal properties, rather
than using explicit details of a given category. We'll give a few universal properties here.
\begin{defn}
Let \(\fC\) be a category.
\begin{itemize}
	\item A \term{final} (or \term{terminal}) object in \(\fC\) is a \(*\in\fC\) such that for all \(X\in\fC\),
	there's a unique map \(X\to*\).
	\item An \term{initial} object is a \(*\in\fC\) such that for all \(X\in\fC\), there's a unique map \(*\to X\).
\end{itemize}
\end{defn}
This is not the last time we'll have dual constructions produced by reversing the arrows.
\begin{exm}
If \(\fC\) is a poset, then a terminal object is exactly a maximum element, and an initial object is a minimum
element. Thus, in particular, they do not necessarily exist.
\end{exm}
Nonetheless, if a final (or initial)  object exists, it's necessarily unique.
\begin{prop}
Let \(*\) and \(*'\) be terminal objects in \(\fC\); then, there's a unique isomorphism \(*\) to \(*'\).
\end{prop}
\begin{proof}
There's a unique map \(*\to *\), which therefore must be the identity, and there are unique maps \(*\to *'\) and
\(*'\to *\), so composing these, we must get the identity, so such an isomorphism exists, and it must be unique,
since there's only one map \(*\to *'\).
\end{proof}
By reversing the arrows, the same thing is true for initial objects. Thus, if such an object exists, it's unique,
so one often hears ``the'' initial or final object. These will be useful for constructing other universal
properties.
\begin{comp}{exm}{enumerate}
	\item In the category of sets, or in the category of topological spaces, the final object is a single point:
	everything maps to the point. The initial object is the empty set, since there's a unique (empty) map to any
	set or space.
	\item In \(\mathsf{Ab}\) or \(\mathsf{Vect}_k\) (abelian groups and vector spaces, respectively), \(0\) is both
	initial and terminal: the unique map is the zero map. An object that is initial and final is called a
	\term{zero object}; as in the case of sets, it may not exist.
	\item In the category of rings, \(0\) is terminal, but not initial (since a map out of \(0\) must send \(0 =
	1\) to \(0\) and \(1\)). \(\Z\) is initial, with the unique map determined by \(1\mapsto 1\).\footnote{That
	rings and ring homomorphisms are unital is important for this to be true.}
	\item Even though we don't really understand what an affine scheme is yet, we know that \(\Spec\Z\) has to be a
	terminal object, and \(\Spec 0\) has to be the initial object. Since we want this to be geometric, then
	\(\Spec\Z\) will play the role of a point. It might not look like a point, but categorically it behaves like
	one.
	\item The category of fields is also interesting: setting \(1 = 0\) isn't allowed, so there are neither initial
	nor terminal objects! If we specialize to fields of a given characteristic, then we get a unique map out of
	\(\Q\) or \(\F_p\), so the category of fields of a given characteristic is initial.
	\item The poset \(\Top(X)\) has \(\varnothing\) initial and \(X\) terminal: it has top and bottom objects.
\end{comp}
The fact that initial and terminal objects are unique means that if you characterize an object in terms of initial
or terminal objects, then you know they're unique as soon as they exist.
\begin{defn}
If \(R\) is a ring, we have the category \(\Alg_R\) of \(R\)-algebras (rings \(T\) with the extra structure of a
map \(R\to T\); morphisms must commute with this map). This is an example of something more general, called an
\term{undercategory}: if \(\fC\) is a category and \(X\in\fC\), then the undercategory \(X\downarrow \fC\) is the
category whose objects are data of \(Y\in\fC\) with \(\fC\)-morphisms \(a_Y: X\to Y\) and whose morphisms are
commutative diagrams
\[\xymatrix@C=0.5cm{
	Y_1\ar[rr]^\vp && Y_2\\
	& X.\ar[ul]^{a_{Y_1}}\ar[ur]_{a_{Y_2}}
}\]
In the same way, the \term{overcategory} \(X\uparrow\fC\) is the same idea, but with maps to \(X\) rather than from
\(X\) (e.g.\ spaces over a given space \(X\)).
\end{defn}
Thus, it's possible to concisely define \(\Alg_R = R\downarrow\mathsf{Ring}\). We will see other examples of this.
\begin{exm}[Localization]
Let \(R\) be a ring and \(S\subset R\) be a multiplicative subset. Then, the \term{localization at \(S\)} is
\(S^{-1}R = \set{r/s\mid r\in R, s\in S: r/s = r/s'\text{ when } s''(rs' - r's) = 0\text{ for some } s''\in S}\).
This is a construction we'll use a lot, so it will be useful to have a canonical characterization of them.

Now, let \(\fC\) be the category of \(R\)-algebras \(T\) with maps \((\vp_T:R\to T\) such that (and this is
a property, not structure) \(\vp_T(s)\) is invertible in \(T\) for all \(s\in S\).
\begin{ex}
Show that \(S^{-1}R\) is the initial object in \(\fC\).
\end{ex}
The naïve idea that localization is ``fractions in \(S\)'' is true if \(R\) is an integral domain, but if
we have zero divisors, the \(R\)-algebra structure map \(R\to S^{-1}R\) need not be injective. But the point is
that if \(T\) is an \(R\)-algebra where the elements of \(S\) become invertible, the map \(\vp_T\) factors through
\(S^{-1}R\); this means that \(S^{-1}R\) is the element of \(\fC\) that's ``closest to \(R\).'' However, you still
have to concretely build it to show that it exists; however, we know already that it's determined up to unique
isomorphism, so we say ``the'' localization.
\end{exm}
Another very fundamental language for making constructions is that of limits and colimits. It may seem a little
strange, but it's quite important.
\begin{defn}
Let \(I\) be a \term{small category} (so its objects form a set); in the context of limits, we will refer to it as
an \term{index category}. Then, a functor \(A:I\to\fC\) is called a \term{\(I\)-shaped} (or \term{\(I\)-indexed})
\term{diagram in \(\fC\)}.
\end{defn}
That is, if \(m:i\to j\) is a morphism in \(I\), then this diagram contains an arrow \(A(m): A_i\to A_j\).
\begin{defn}
Let \(A\) be an \(I\)-shaped diagram in \(\fC\). Then, a \term{cone} on \(A\) is the data of an object \(B\in\fC\)
and maps \(A_i\to B\) for every \(i\in I\) commuting with the morphisms in \(I\).
\begin{figure}[h!]
\[\xymatrix{
	A_1\ar[r]\ar[d]\ar@/_0.3cm/[drrr] & A_3\ar[drr]\\
	A_2\ar[ur]\dblarrow[r][f][g] &&& B
}\]
\caption{A cone on a diagram \(A\).}
\end{figure}
The cones on \(A\) form a category \(\mathsf{Cones}_A\), where the morphisms are maps \(B\to B'\) commuting with
all the maps in the cone.
\end{defn}
We can also take the category of ``co-cones,'' which are data of maps from \(B\) \emph{into} the diagram. This is
not quite the opposite category (since we want maps \(B\to B'\) commuting with the maps into the
diagram).\footnote{Some people switch the definitions of cones and co-cones, but since we're not going to use these
words very much, it doesn't matter all that much.}
\begin{comp}{defn}{itemize}
	\item The \term{colimit} \(\varinjlim_I A\) is the initial object in the category of cones of \(A\).
	\item The \term{limit} \(\varprojlim_I A\) is the terminal object in the category of co-cones of \(A\).
\end{comp}
As before, colimits and limits may or may not exist, but if they do, they're unique up to unique isomorphism.

Colimits act like a quotient, and it's easier to map out of them. Correspondingly, limits behave like a subobject,
and it's easier to map into them.
\begin{exm}[Products and Coproducts]
Let \(I = \bullet\ \bullet\) be a two-element discrete set (no non-identity arrows). Thus, an \(I\)-shaped diagram
is just a choice of two spaces \(A_1\) and \(A_2\), so a colimit \(C_1\) is the data of a unique map \(\vp_B\) for
each \(B\in\fC\) fitting into the following diagram.
\[\xymatrix@R=0.3cm{
	A_1\ar[dr]\ar@/^0.4cm/[drr]\\
	& C_1\ar[r]^{\vp_B} & B\\
	A_2\ar[ur]\ar@/_0.4cm/[urr]
}\]
This is called the \term{coproduct} of \(A_1\) and \(A_2\), denoted \(A_1\sqcup A_2\) or \(A_1\amalg A_2\).

Similarly, the limit of \(A\) is called the \term{product} of \(A_1\) and \(A_2\), is denoted \(A_1\times A_2\),
and fits into the diagram
\[\xymatrix@R=0.3cm{
	&& A_1\\
	B\ar[r]\ar[urr]\ar[drr] & A_1\times A_2\ar[ur]\ar[dr]\\
	&& A_2
}\]
In the same way, if \(I\) is a larger discrete set, we get coproducts and products of objects in \(\fC\) indexed by
\(I\), denoted \(\coprod_I A_i\) and \(\prod_I A_i\), respectively.

In the category of sets, the product is Cartesian product, and the coproduct is disjoint union. The same is true in
topological spaces.

In the category of groups, the product is once again Cartesian product, but the coproduct is the free product
(mapping out of it is the same as mapping out of the individual components, which is not true of the direct
product). As underlying sets, this is distinct from the coproduct of sets.

In linear categories, e.g.\ \(\mathsf{Ab}\), \(\mathsf{Mod}_R\), or \(\mathsf{Vect}_k\), \(V\oplus W\) is the
product and coproduct, and the same is true over all finite \(I\). However, this is \emph{not} true when \(I\) is
infinite: the coproduct is the direct sum, which takes finite sums of elements, and the product is the Cartesian
product, which takes arbitrary sums of elements. It's worth working out why this is, and how it works.
\end{exm}
Many of these categories are ``sets with structure,'' e.g.\ groups, vector spaces, topological spaces, and so on.
In these cases, there is a \term{forgetful functor} which forgets this structure: indeed, a group homomorphisms
(continuous map, linear map) is a map of sets too.\footnote{If this seems vague, that's all right; it's possible to
define and find forgetful functors more formally.}

There's a useful principle here: \emph{forgetful functors preserve limits}: if \(F\) is a forgetful functor, then
there is a canonical isomorphism \(F(\varprojlim A)\cong\varprojlim F(A)\). This is something that can be defined
more rigorously and proven. But one important corollary is that if you know what the limit looks like for sets,
it's the same in groups, rings, vector spaces, topological spaces, and so on. However, this is very false for
coproducts, e.g.\ the coproduct on groups is not the same as the one on sets.

This becomes a little cooler once we see limits that aren't just products.
\begin{exm}
Consider the diagram of rings
\[\xymatrix{
\dotsb\ar[r] & \Z/p^n\ar[r] & \dotsb\ar[r] & \Z/p^3\ar[r] & \Z/p^2\ar[r] & \Z/p,
}\]
where each map is given by modding out by \(p\). One can show that the limit exists, and it'll be the same as
the limit of the underlying sets, a sequence of compatible elements; this limit is called the \term{\(p\)-adic
integers}, denoted \(\Z_p\). More generally, the same thing works for \(\varprojlim R/I^n\) for an ideal \(I\subset
R\), and defines the \term{\(I\)-adic completion} \(\widehat R_I\), which we'll revisit, since it has useful
geometric meaning.
\end{exm}
